\documentclass{article}
\usepackage[utf8]{inputenc}
\usepackage{graphicx}
\usepackage[affil-it]{authblk}
\usepackage{amsmath}
\usepackage{tabularx}
\usepackage{hyperref}
\usepackage{float}
\graphicspath{{../img/}}

\title{\textbf{Spatiotemporal Trends in Pacific Northwest Wildfire Occurrence and Firefighting Resource Demand}}
\author{Blaise Rettig}
\affil{Fowler School of Engineering, Chapman University} 
\date{08 December 2025}

\begin{document}
\maketitle

\section{Background and Research Objectives}

    Wildfires are a growing concern across the Pacific Northwest (PNW), where rising temperatures, prolonged droughts, and increasing human development along the wildland–urban interface have amplified both the frequency and severity of fire events. Studies in recent years have documented a strong correlation between climate variability and wildfire occurrence, while others (such as those published in Nature Communications and Ecosphere) highlight the compounding social and infrastructural impacts that follow major fire seasons. In the PNW, the interaction between regional climate patterns, vegetation dynamics, and human settlement has produced increasingly complex fire regimes that challenge both ecological stability and resource management.
    
    Wildfire research has traditionally focused on burn severity, vegetation recovery, or climate drivers. However, fewer studies have developed integrated models that connect wildfire occurrence to infrastructure and environmental consequences, such as changes in energy production, water quality, or firefighting resource demand. Understanding these cascading effects is critical for anticipating the full impact of large fire events and for improving strategic resource allocation during response efforts.
    
    The objective of this research is to identify and model spatiotemporal trends in wildfire occurrence and firefighting resource demand across the Pacific Northwest, and to build a predictive framework that estimates downstream environmental and infrastructural effects of new wildfire events. Specifically, the study seeks to:
    
    \begin{itemize}
        \item Analyze temporal and spatial patterns in wildfire occurrence from 2002–2024 using historical satellite and incident data.
        \item Model relationships between wildfire location, size, and intensity with key environmental and infrastructural factors such as energy output, population displacement, and water system proximity.
        \item Develop a Python-based predictive model capable of estimating indices for potential impact severity (e.g., energy disruption index, displacement risk index, resource demand index) when given a new wildfire location.
        \item Visualize trends and predictive outputs using QGIS to support regional planning and firefighting decision-making.
    \end{itemize}
    
    This research aims to bridge data science and environmental management by integrating time-series analysis, spatial modeling, and supervised learning to quantify and forecast wildfire impacts within the PNW.
    
    \section{Sources of Data and Description of Analyses/Methods}
    
    This research integrates multiple datasets to examine and model the relationships among wildfire occurrence, firefighting resource demand, and secondary environmental and infrastructural impacts in the Pacific Northwest (PNW). The framework combines temporal modeling, spatial analysis, and predictive modeling using open-access geospatial and tabular data. The selection of data sources and analytical methods is informed by previous studies that have quantified wildfire impacts on solar energy generation, power grids, hydrological systems, atmospheric processes, and community vulnerability.
    
    \subsection{Data Sources}

    \subsubsection{Wildfire Activity and Severity}
    Historical wildfire activity and severity will be characterized using datasets from the Monitoring Trends in Burn Severity (MTBS), which provides fire perimeters, burn severity indices, and postfire vegetation metrics across the Pacific Northwest. Additional operational information, including resource mobilization, fire duration, personnel deployment, and suppression costs, will be derived from National Interagency Fire Center (NIFC) records and ICS-209 incident reports.

    \subsubsection{Environmental and Climatic Variables}
    Environmental and climatic drivers of wildfire behavior will be assessed using temperature and precipitation data from the PRISM Climate Group, wind speed estimates from the TerraClimate reanalysis product, and smoke-related PM\textsubscript{2.5} exposure data from the Stanford EchoLab. Together, these sources capture both long-term conditions and acute atmospheric impacts relevant to wildfire dynamics.

    \subsubsection{Infrastructure and Energy Systems}
    Impacts of wildfires on critical energy infrastructure will be examined through datasets from the U.S. Energy Information Administration (EIA), which provide information on electricity generation and total energy output. These data enable evaluation of sectoral disruptions linked to wildfire events.

    \subsubsection{Socioeconomic and Community Vulnerability}
    Societal exposure, vulnerability, and potential displacement will be analyzed using population density metrics and the Social Vulnerability Index (SVI) from the U.S. Census Bureau and FEMA. These sources support assessment of how wildfire hazards intersect with community characteristics.\newline

    Together, these datasets form a comprehensive foundation for analyzing the interactions among wildfire occurrence, firefighting resource demand, and multi-sector impacts across the Pacific Northwest.

    \subsection{Analytical Methods}
    
    \subsubsection{Temporal Analysis}
    Python-based time series analysis will be used to explore changes in wildfire frequency, burned area, and resource demand over time.
    \paragraph{Libraries.} pandas, geopandas, numpy, rasterstats, matplotlib
    \paragraph{Methods.} Trend decomposition, forecasting, and correlation analysis between climate anomalies, ICS-209 information, energy data, and fire activity.
    
    \subsubsection{Spatial Analysis}
    Spatial modeling will be conducted in QGIS to visualize and quantify clustering of wildfire activity:
    \begin{itemize}
        \item Hotspot detection.
        \item Raster overlay analysis to assess the proximity of fires to energy infrastructure.
        \item Various atmospheric difference maps.
    \end{itemize}
    
    \subsubsection{Predictive Modeling}
    A supervised machine learning model will be trained to estimate likely downstream impacts of new wildfire events.
    \paragraph{Framework.} scikit-learn
    \paragraph{Inputs.} Fire location, drought index, proximity to infrastructure, atmospheric conditions, SVI.
    \paragraph{Outputs.} Normalized indices representing predicted severity in five categories — (1) fire acreage, (2) firefighting resource demand, (3) energy disruption, (4) atmospheric/smoke impact, and (5) displacement risk.

\section{Results}

    \subsection{Temporal Results}

        Various interesting figures were obtained through temporal analysis of various datasets.

        \begin{center}
            \includegraphics[width=\linewidth]{1Figure_1.png}
            Figure 1: fire count and total burned area versus year. Data source: MTBS.
        \end{center}

        \begin{center}
            \includegraphics[width=\linewidth]{1Figure_23.png}
            Figure 2: fire size as a function of mean annual temperature and mean annual precipitation. Data source: MTBS, PRISM.
        \end{center}

        \begin{center}
            \includegraphics[width=\linewidth]{1Figure_6.png}
            Figure 6: count of areas that experience frequent re-burn. Data source: MTBS. 
            \begin{verbatim}
            Reburned areas: 929
            \end{verbatim}
        \end{center}

        \begin{center}
            \includegraphics[width=\linewidth]{2Figure_1.png}
            Figure 9: firefighting personnel versus fire size. Data source: ICS-209 via U.S. Forest Service.
        \end{center}

        \begin{center}
            \includegraphics[width=\linewidth]{2Figure_2.png}
            Figure 10: firefighting personnel and cost versus structure risk. Data source: MTBS, ICS-209 via U.S. Forest Service.
            \begin{verbatim}
            Median personnel (structures at risk): 46.8
            Median personnel (no structures): 26.1
            \end{verbatim}
        \end{center}

        \begin{center}
            \includegraphics[width=\linewidth]{2Figure_3.png}
            Figure 11: various analyses relating to firefighting personnel and evacuation risk. Data source: MTBS, ICS-209 via U.S. Forest Service.
            \begin{verbatim}
            Total fires with evacuations: 143
            Total fires with structure loss: 481
            Average loss rate when structures threatened: 7.73%
            \end{verbatim}
        \end{center}

        \begin{center}
            \includegraphics[width=\linewidth]{2Figure_5_1.png}
            Figure 12: total personnel and fires per year. Data source: MTBS, ICS-209 via U.S. Forest Service.
        \end{center}

        \begin{center}
            \includegraphics[width=\linewidth]{3Figure_1.png}
            Figure 13: fire size as affected by wind speed. Data source: MTBS, TerraClimate.
            \begin{verbatim}
                          tmean       ppt  wind_speed   area_ha
            tmean       1.000000 -0.445637    0.281199 -0.017428
            ppt        -0.445637  1.000000   -0.245124  0.053166
            wind_speed  0.281199 -0.245124    1.000000  0.004085
            area_ha    -0.017428  0.053166    0.004085  1.000000
            \end{verbatim}
        \end{center}    

        \begin{center}
            \includegraphics[width=\linewidth]{4Figure_2.png}
            Figure 16: fire activity versus proximity to energy infrastructure. Data source: MTBS, EIA.
            \begin{verbatim}
                                 count   mean   std     min   25%   50%    75%    max
            proximity_category
            Very Close (<10km)    545.0  6790.  15961.  408.  748.  1490.  4649.  200444.
            Close (10-50km)      1225.0  5308.  13755.  405.  700.  1427.  4016.  229622.
            Moderate (50-100km)   451.0  4590.  13891.  402.  757.  1421.  3499.  225660.
            Far (>100km)           63.0  3371.   5310.  416.  648.  1453.  4178.   27459.
            \end{verbatim}
        \end{center}

        \begin{center}
            \includegraphics[width=\linewidth]{5Figure_2.png}
            Figure 18: social vulnerability index versus fire activity. Data source: MTBS, CDC ASTDR.
            \begin{verbatim}
            Top 10 Most Vulnerable Counties:
                    STATE         COUNTY  RPL_THEMES
                    OR    Malheur County      0.9908
                    WA      Adams County      0.9800
                    WA     Yakima County      0.9647
                    WA   Okanogan County      0.9453
                    WA      Grant County      0.9310
                    OR  Jefferson County      0.9278
                    OR   Umatilla County      0.9259
                    WA   Franklin County      0.9147
                    ID Washington County      0.9093
                    OR     Marion County      0.9020
            \end{verbatim}
        \end{center}

        \begin{center}
            \includegraphics[width=\linewidth]{5Figure_4.png}
            Figure 20: fire activity and smoke pollution over time. Data source: MTBS, EchoLab via Stanford University.
        \end{center}

        \begin{center}
            \includegraphics[width=\linewidth]{5Figure_5.png}
            Figure 21: social vulnerability versus smoke exposure. Data source: CDC ASTDR, EchoLab via Stanford University.
        \end{center}

    \subsection{Spatial Results}

        Spatial analysis performed in QGIS resulted in the following figures.

        \begin{center}
            \includegraphics[width=\linewidth]{6Figure_1.png}
            Figure 22: 1984-2024 daily fire occurrences (dots) and fire boundaries (polygons). Data source: MTBS. 
        \end{center}

        \begin{center}
            \includegraphics[width=\linewidth]{6Figure_2.png}
            Figure 23: annual mean temperature difference map between 2024 and 2000.
            \begin{verbatim}
            band_min (blue)     band_max (red)      unit
            -2.91               3.41                C
            \end{verbatim}
        \end{center}

        \begin{center}
            \includegraphics[width=\linewidth]{6Figure_3.png}
            Figure 24: annual mean precipitation difference map between 2024 and 2000.
            \begin{verbatim}
            band_min (gold)     band_max (blue)      unit
            -584                2173                 mm
            \end{verbatim}
        \end{center}

        \begin{center}
            \includegraphics[width=\linewidth]{6Figure_4.png}
            Figure 25: wind speed visualization in low-fire year (2024).
        \end{center}

        \begin{center}
            \includegraphics[width=\linewidth]{6Figure_5.png}
            Figure 26: wind speed visualization in high-fire year (2000).
        \end{center}

        \begin{center}
            \includegraphics[width=\linewidth]{6Figure_6.png}
            Figure 27: heatmap (kernel density estimation) of fire activity.
        \end{center}

        \begin{center}
            \includegraphics[width=\linewidth]{6Figure_7.png}
            Figure 28: heatmap (kernel density estimation) of fire activity with boundaries and occurrences overlayed.
        \end{center}

        \begin{center}
            \includegraphics[width=\linewidth]{6Figure_8.png}
            Figure 29: power plants and energy producing infrastructure (yellow dots).
        \end{center}

        \begin{center}
            \includegraphics[width=\linewidth]{6Figure_9.png}
            Figure 30: the six most socially vulnerable counties across each state.
            \begin{verbatim}
            STATE            COUNTY 
            OR       Malheur County
            OR     Jefferson County
            OR      Umatilla County 
            OR        Marion County
            OR         Wasco County
            OR          Lake County
            
            WA         Adams County
            WA        Yakima County
            WA      Okanogan County
            WA         Grant County
            WA      Franklin County
            WA  Grays Harbor County
            
            ID    Washington County
            ID        Elmore County
            ID       Gooding County
            ID         Power County
            ID      Shoshone County
            ID        Owyhee County

            \end{verbatim}
        \end{center}

    \subsection{Predictive Modeling Results}

        Using all of the above mentioned data, a predictive model was developed and trained to estimate the downstream impacts of wildfires.

        The model creates four impact 'indices' for wildfire events: resource demand (firefighting personnel), evacuation risk (displaced persons), structure threat (buildings at risk), and suppression costs.

        The system employs Gradient Boosting Regressors for each target variable:
        \begin{itemize}
            \item \textbf{Algorithm}: Gradient Boosting with 150 estimators, learning rate 0.05, max depth 6
            \item \textbf{Target transformation}: Log-transform $y' = \log(1 + y)$ to handle skewed distributions
            \item \textbf{Feature scaling}: StandardScaler for input features, MinMaxScaler for output normalization
        \end{itemize}

        For each wildfire scenario, the model returns:
        \begin{align*}
        \text{Resource Demand} &: \text{Firefighting personnel needed} \\
        \text{Evacuation Risk} &: \text{Number of people displaced} \\
        \text{Structure Threat} &: \text{Buildings threatened or destroyed} \\
        \text{Suppression Cost} &: \text{Estimated incident management costs}
        \end{align*}

        \paragraph{Example Training Output.}

        \begin{verbatim}
        ============================================================
        TRAINING
        ============================================================
        
        --- Training model for: resource_demand ---
          Valid training samples: 1540 / 1567
          Valid test samples: 389 / 392
          MAE: 161.02
          RMSE: 268.86
          R^2 (raw): 0.189
          R^2 (log scale): 0.237
          Mean actual: 242.98
          Mean predicted: 161.76
        
        --- Training model for: evacuation_risk ---
          Valid training samples: 86 / 1567
          Valid test samples: 26 / 392
          MAE: 150.94
          RMSE: 299.97
          R^2 (raw): 0.060
          R^2 (log scale): 0.253
          Mean actual: 220.62
          Mean predicted: 156.83
        
        --- Training model for: structure_threat ---
          Valid training samples: 1019 / 1567
          Valid test samples: 249 / 392
          MAE: 221.31
          RMSE: 981.05
          R^2 (raw): 0.685
          R^2 (log scale): 0.286
          Mean actual: 306.98
          Mean predicted: 140.90
        
        --- Training model for: suppression_cost ---
          Valid training samples: 1270 / 1567
          Valid test samples: 311 / 392
          MAE: 2770201.82
          RMSE: 5348756.80
          R^2 (raw): 0.253
          R^2 (log scale): 0.219
          Mean actual: 3555619.55
          Mean predicted: 2536414.66

        ============================================================
        EXAMPLE PREDICTIONS
        ============================================================
        
        --- Small Fire (100 ha) ---
        Raw Predictions:
          Personnel needed: 14
          People evacuated: 87
          Structures threatened: 22
          Suppression cost: $1,069,721
        
        Normalized Indices (0-1) * 10:
          Resource demand: 0.02
          Evacuation risk: 0.02
          Structure threat: 0.00
          Suppression cost: 0.06
        
        --- Large Fire (10,000 ha) ---
        Raw Predictions:
          Personnel needed: 538
          People evacuated: 133
          Structures threatened: 319
          Suppression cost: $4,949,684
        
        Normalized Indices (0-1) * 10:
          Resource demand: 0.78
          Evacuation risk: 0.03
          Structure threat: 0.02
          Suppression cost: 0.09
        \end{verbatim}

        \begin{center}
            \includegraphics[width=\linewidth]{model_performance_improved.png}
            Figure 31: performance statistics.
        \end{center}

        \begin{center}
            \includegraphics[width=\linewidth]{prediction_examples_improved.png}
            Figure 32: graphical predictions.
        \end{center}

\section{Discussion and Conclusions}

    \subsection{Predictive Model Discussion}

    As is obvious, the predictive model is quite bad. In terms of suppression cost, resource demand, and evacuation risk, the model only explains 25.3\%, 18.9\%, and 6.0\% variation in the dependent variable respectively. The only prediction it makes that is marginally better than picking the average value every time is suppression cost, with a coefficient of determination of 68.5\%.

    These sub-par metrics indicate that wildfire impact prediction is an inherently complex problem. Several factors likely contribute to the model's limited explanatory power. First, operational firefighting decisions involve numerous real-time decisions, political pressures, and resource availability constraints that cannot be captured through historical climate and spatial data alone. Second, evacuation orders depend heavily on local emergency management protocols, community preparedness levels, and perceived threat rather than purely physical fire characteristics. Third, the high variability in suppression costs reflects not just fire size and behavior, but also jurisdictional differences in accounting and suppression versus managed fire decisions.

    \subsection{Spatiotemporal Trends and Climate Relationships}

    The temporal analyses reveal several trends. Figure 1 demonstrates the relationship between fire count per year and total burned area. The large peaks in both graphs around 2006 and 2012 suggest that the large number of fires overwhelmed firefighting efforts leading to large burn areas, while the other large peaks on the fire count graph, 2000, 2017, etc., are not substantiated by similarly large peaks on the burn area graph, indicating that these fires were effectively combated.

    Climate variable relationships proved more nuanced than anticipated. Figure 2 shows a weak negative correlation between mean annual temperature and fire size (contrary to expectations). The precipitation relationship (Figure 3) similarly shows high variability, suggesting that simple climate metrics alone are insufficient predictors beyond what is obvious (lower precipitation and higher temperature indicate more fires).

    The drought analysis (Figure 5) revealed no statistically significant temperature difference (p=0.437) between drought and non-drought conditions during fire events. The Random Forest feature importance analysis (Figure 4) confirmed that temperature and precipitation values are stronger predictors of fire size than drought indices.

    Fire season length has generally expanded (Figure 7), with the number of active fire days per year increasing over the period. The identification of 929 reburned areas (Figure 6) highlights the growing prevalence of short-interval reburns.

    \subsection{Infrastructure and Societal Impacts}

    The analysis of firefighting resource deployment (Figures 9-12) shows relationships between structural threat and personnel allocation. Fires threatening structures required nearly double the median personnel (46.8 vs. 26.1) compared to backcountry fires. However, the 7.73\% average loss rate when structures were threatened suggests that even with more resource deployment, structural protection remains challenging during major fire events.

    Energy infrastructure analysis (Figures 15-17) showed spatial correlations between fire activity and power plants, with fires occurring most frequently in the "Close (10-50km)" category. However, the relationship between major fire years and electricity generation patterns was inconsistent, suggesting that while fires may cause localized disruptions, the regional grid is resilient.

    Social vulnerability mapping (Figures 18, 21, 30) identified disparities in wildfire exposure across the PNW. Counties with the highest social vulnerability indices consistently experience elevated fire exposure and smoke pollution.

    Smoke exposure analysis (Figures 20-21) demonstrated that PM2.5 concentrations have increased substantially during major fire years.

    \subsection{Spatial Patterns and Hotspot Identification}

    The spatial analysis conducted in QGIS (Figures 22-30) shows geographic patterns in fire occurrence. The kernel density estimation (Figures 27-28) identified persistent fire hotspots in eastern Oregon, central Washington, and southern Idaho, regions characterized by dry forests, shrublands, and grasslands with high fuel loads. These hotspots align with areas experiencing the greatest temperature increases (Figure 23) and precipitation declines (Figure 24) over the study period.

    Wind speed visualizations (Figures 25-26) comparing low-fire year 2024 with high-fire year 2000 showed relatively modest differences in mean wind patterns, suggesting that perhaps extreme wind events rather than seasonal averages may be more critical in fire spread. This indicates that incorporating wind gust frequency and direction during fire-weather events could improve the predictive model's performance.

    \subsection{Comparison to Existing Literature}

    These findings both confirm and extend previous wildfire research. The expansion of fire season length and increasing burned area align with projections from climate-fire models published in Nature and other journals.

    The disparities in smoke exposure across socioeconomic levels support recent environmental justice research highlighting how climate impacts can further inequality.

    The limitations of predictive modeling observed here support the challenges reported in more well-established fire behavior and prediction systems. Even with extensive data integration, the stochastic parts of fire behavior and human decision-making variability remain difficult to capture in statistical models.

    \subsection{Implications for Management and Further Research}

    Despite the predictive model's limitations, this research still provides valuable insight for wildfire management and resource allocation. The spatial hotspot analysis can help with pre-positioning of firefighting assets, while the social vulnerability mapping identifies communities that require more preparedness and evacuation planning.
    
    Future research should focus on several key areas. First, incorporating more high-resolution fire weather indices and real-time fuel or moisture measurements could substantially improve the predictive accuracy. Second, integrating suppression strategy data (such as full suppression vs. managed fire) would help resource demand modeling. Third, extending the analysis to include post-fire recovery and economic impacts would provide a better picture of wildfire consequences.

    Additionally, the model could benefit from machine learning approaches that better handle non-linear relationships and dependencies, such as recurrent neural networks.
    
    \subsection{Conclusions}

    This research demonstrates that Pacific Northwest wildfires exhibit clear spatiotemporal trends, with increasing frequency, longer fire seasons, and expanding burned area over the 1984-2024 period. Climate variables show large relationships with fire size, and significant disparities exist in how wildfire impacts are distributed across communities of varying socioeconomic status.

    The predictive model reveals some fundamental challenges in forecasting wildfire impacts from environmental and spatial data alone. While structure threat can be reasonably estimated from spatial proximity and fire characteristics, human dimensions such as evacuation decisions and personnel deployment remain largely unpredictable.
    
    Ultimately, this work contributes to the growing body of evidence that wildfire management must integrate climate data, spatial analysis, and social considerations to effectively protect communities and ecosystems.

\section{Statement of AI Usage}

    AI models including ChatGPT, Claude, and Cursor were used in this project in the following ways:
    \begin{itemize}
        \item Brainstorming and ideation on research topic/question.
        \item Inquiring for relevant datasets.
        \item Inquiring for general workflow advice.
        \item Inquiring for QGIS raster and vector tools.
        \item Inquiring for interesting and novel ways to visualize temporal data.
        \item Prompting for code generation for boilerplate tasks, bugfixing, and specific temporal  analyses (social vulnerability and PM2.5 smoke).
    \end{itemize}

\section{References}

\noindent Abatzoglou, J. T., \& Williams, A. P. (2016). Impact of anthropogenic climate change on wildfire across western US forests. \textit{Proceedings of the National Academy of Sciences}, 113(42), 11770-11775.

\noindent Burke, M., Driscoll, A., Heft-Neal, S., Xue, J., Burney, J., \& Wara, M. (2021). The changing risk and burden of wildfire in the United States. \textit{Proceedings of the National Academy of Sciences}, 118(2), e2011048118.

\noindent Corwin, K. A., Burkhardt, J., Corr, C. A., Stackhouse, P. W., Munshi, A., \& Fischer, E. V. (2025, January 2). \textit{Solar Energy Resource Availability under extreme and historical wildfire smoke conditions.} Nature News.

\noindent Dennison, P. E., Brewer, S. C., Arnold, J. D., \& Moritz, M. A. (2014). Large wildfire trends in the western United States, 1984–2011. \textit{Geophysical Research Letters}, 41(8), 2928-2933.

\noindent Kolden, C. A. (2019). We're not doing enough prescribed fire in the Western United States to mitigate wildfire risk. \textit{Fire}, 2(2), 30.

\noindent Parks, S. A., Miller, C., Abatzoglou, J. T., Holsinger, L. M., Parisien, M. A., \& Dobrowski, S. Z. (2016). How will climate change affect wildland fire severity in the western US? \textit{Environmental Research Letters}, 11(3), 035002.

\noindent Vahedi, S., Zhao, J., Pierre, B., Lei, F., Anagnostou, E., He, K., Jones, C., \& Wang, B. (2025, March 24). \textit{Wildfire and Power Grid Nexus in a changing climate.} Nature News. 

\noindent Westerling, A. L. (2016). Increasing western US forest wildfire activity: sensitivity to changes in the timing of spring. \textit{Philosophical Transactions of the Royal Society B}, 371(1696), 20150178.

\section{Appendix}

\subsection{Additional Figures}

    \begin{center}
        \includegraphics[width=\linewidth]{1Figure_7.png}
        Figure 7: fire season length and number of wildfires plotted over time. Data source: MTBS.
    \end{center}

    \begin{center}
        \includegraphics[width=\linewidth]{1Figure_8.png}
        Figure 8: various drought related analyses. Data source: MTBS, U.S. Drought Monitor via UNL. 
    \end{center}

    \begin{center}
        \includegraphics[width=\linewidth]{4Figure_1.png}
        Figure 15: electricity generation versus fire activity. Data source: MTBS, EIA.
    \end{center}

\subsection{Code Repository}

All analysis code, scripts, and QGIS files are available in the project GitHub repository:

\begin{center}
\texttt{https://github.com/blaiserettig/predicting-fire}
\end{center}

\subsection{Data Sources and Access Information}

See Table 1.

\begin{table}[ht]
\centering
\caption{Data Sources and Access Information}
\renewcommand{\arraystretch}{1.3}
\begin{tabularx}{\textwidth}{p{3cm} p{2.5cm} X p{3cm}}
\hline
\textbf{Variables / Dataset} & \textbf{Data Type} & \textbf{Input Data / Description} & \textbf{Source / Access} \\
\hline

MTBS (Monitoring Trends in Burn Severity) &
Observational &
Fire perimeters and burn severity indices &
\href{https://www.mtbs.gov/direct-download}{MTBS Download Portal} \\

ICS-209 Incident Reports &
Observational &
Firefighting resource deployment and suppression costs; Incident summaries &
\href{https://research.fs.usda.gov/firelab/products/dataandtools/ics-209-plus}{USFS Fire Modeling Institute} \\

U.S. Drought Monitor (DSCI) &
Modeled, observational &
Weekly drought severity classifications &
\href{https://droughtmonitor.unl.edu/DmData/DataDownload/DSCI.aspx}{U.S. Drought Monitor DSCI} \\

PRISM Climate Data &
Observational &
Temperature and precipitation time series &
\href{https://prism.oregonstate.edu/data/}{PRISM Climate Group} \\

TerraClimate &
Modeled &
Wind speed and environmental climate variables &
\href{https://www.climatologylab.org/terraclimate.html}{TerraClimate} \\

Stanford EchoLab Wildfire Smoke Dataset &
Modeled / Observational &
Daily PM\textsubscript{2.5} concentrations attributable to wildfire smoke &
\href{https://www.stanfordecholab.com/wildfire_smoke}{Stanford EchoLab Smoke Data} \\

U.S. EIA Power Plant Data &
Observational &
Electricity generation and power plant locations &
\href{https://www.eia.gov/opendata/browser/}{U.S. EIA Open Data} \\

CDC/ATSDR Social Vulnerability Index (SVI) &
Observational &
County-level social vulnerability indicators &
\href{https://www.atsdr.cdc.gov/place-health/php/svi/svi-data-documentation-download.html}{CDC / ATSDR SVI} \\
\hline
\end{tabularx}
\end{table}


\subsection{Software and Tools}

\subsubsection{Geospatial Analysis}
\begin{itemize}
    \item \textbf{QGIS 3.x} -- Spatial visualization, kernel density estimation, raster overlay analysis
    \item \textbf{Python geopandas} -- Vector data processing and spatial joins
    \item \textbf{Python rasterstats} -- Raster value extraction and zonal statistics
\end{itemize}

\subsubsection{Data Analysis and Modeling}
\begin{itemize}
    \item \textbf{Python pandas/numpy} -- Data manipulation and numerical computation
    \item \textbf{Python scikit-learn} -- Machine learning model development (Gradient Boosting Regressor, Random Forest)
    \item \textbf{Python matplotlib/seaborn} -- Statistical visualization and plotting
\end{itemize}

\subsection{Study Area Definition}

The Pacific Northwest study region encompasses:
\begin{itemize}
    \item \textbf{Oregon} -- All counties
    \item \textbf{Washington} -- All counties  
    \item \textbf{Idaho} -- All counties
\end{itemize}

Total area: approximately 568,000 km$^2$ \\
Study period: 1984--2024 (limited by MTBS data availability) \\
Temporal resolution: Annual aggregation for trend analysis; daily for event-level modeling

\end{document}